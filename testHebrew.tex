\documentclass[12pt]{article}



\usepackage[colorlinks]{hyperref}



\title{\textenglish{Using Hebrew Fonts on Overleaf with polyglossia}\\דוגמא לגופני ברירת מחדל}
\author{\textenglish{Overleaf}}

\begin{document}

\maketitle

\section{דוגמא פשוטה}
אבגדהוזחטיךכלםמןנסעףפץצקרשת \textbf{בולט} \emph{מודגש} \textsf{פונט סאנס סריף} \texttt{רוחב קבוע}

\begin{english}
Remember to set your project's compiler to XeLaTeX or LuaLaTeX. For more details see \href{https://www.overleaf.com/learn/latex/Multilingual_typesetting_on_Overleaf_using_polyglossia_and_fontspec}{this help page}. Hebrew typefaces available on Overleaf are listed on \href{https://www.overleaf.com/learn/latex/Questions/Which_OTF_or_TTF_fonts_are_supported_via_fontspec%3F#Fonts_for_Hebrew_script}{this page}.
\end{english}

\end{document}
